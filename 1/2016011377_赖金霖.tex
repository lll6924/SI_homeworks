\documentclass{article} 
\usepackage{ctex}  
\usepackage{amsfonts}
\usepackage{amsmath}
\usepackage{amssymb}
\title{Theory and Methods for Statistical Inference - Homework 1}
\author{CST65 赖金霖 2016011377}   
\date{\today}  

\usepackage[a4paper,left=10mm,right=10mm,top=15mm,bottom=15mm]{geometry}  
\begin{document}
\maketitle    
\section{binary communication system}
\paragraph{(a)}
The value range of r is $[-\frac{3}{4},\frac{7}{4}]$. 
The probability-of-error of a detector $\hat{m}(r)$ is
\begin{align}
\mathbb{P}(\hat{m}(r)\neq m) &= \mathbb{P}(\hat{m}(r)=0| m=1)\mathbb{P}(m=1)+\mathbb{P}(\hat{m}(r)=1| m=0)\mathbb{P}(m=0) \notag\\
&= \frac{3}{4}\int^{\frac{7}{4}}_{\frac{1}{4}}\frac{2}{3}\mathbb{I}[\hat{m}(r)=0]dr+\frac{1}{4}\int^{\frac{3}{4}}_{-\frac{3}{4}}\frac{2}{3}\mathbb{I}[\hat{m}(r)=1]dr\notag\\
&=\frac{1}{2}\int^{\frac{7}{4}}_{\frac{3}{4}}\mathbb{I}[\hat{m}(r)=0]dr+\frac{1}{6}\int^{\frac{1}{4}}_{-\frac{3}{4}}\mathbb{I}[\hat{m}(r)=1]dr+\frac{1}{6}\int^{\frac{3}{4}}_{\frac{1}{4}}(3\mathbb{I}[\hat{m}(r)=0]+\mathbb{I}[\hat{m}(r)=1])dr\notag
\end{align}
Obviously, the three elements of the above equation is non-negative and can be considered separately. Therefore, the minimum probability-of-error detector $\hat{m}(r)$ should be 0 in $[-\frac{3}{4},\frac{1}{4}]$ and 1 in $[\frac{3}{4},\frac{7}{4}]$. As for $[\frac{1}{4},\frac{3}{4}]$, we should set $\hat{m}(r)$ to 1 because the coefficient of $\mathbb{I}[\hat{m}(r)=1]$ is three times lower than $\mathbb{I}[\hat{m}(r)=0]$. In this case, $\hat{m}(r)$ satisfies
\begin{align}
	\hat{m}(r)=\begin{cases}
		0 & -\frac{3}{4}\le r\le \frac{1}{4}\\
		1 & \frac{1}{4}\le r\le \frac{7}{4}
	\end{cases}\notag
\end{align}
The associated probability of error is therefore $\frac{1}{12}$.

\paragraph{(b)}
If the receiver does not know the prior probabilities and has to use a ML detector, it would compare $p(r|m=0)$ and $p(r|m=1)$, which are
\begin{align}
	p(r|m=0)&=\begin{cases}
		\frac{2}{3} & -\frac{3}{4}\le r\le \frac{3}{4}\\
		0 & otherwise
	\end{cases}\notag \\
	p(r|m=1)&=\begin{cases}
		\frac{2}{3} & \frac{1}{4}\le r\le \frac{7}{4}\\
		0 & otherwise
	\end{cases}\notag
\end{align}
In $[-\frac{3}{4},\frac{1}{4}]$, the detector must select 0, while in $[\frac{3}{4},\frac{7}{4}]$, 1 is preferred. In $[\frac{1}{4},\frac{3}{4}]$, we could choose either 0 or 1 for the detector. So there should be infinitely many ML detectors. We can therefore design two different ML detectors, $\hat{m}_1(r)$ and $\hat{m}_2(r)$, which are
\begin{align}
	\hat{m}_1(r)&=\begin{cases}
		0 & -\frac{3}{4}\le r\le \frac{1}{4}\\
		1 & \frac{1}{4}\le r\le \frac{7}{4}
	\end{cases}\notag \\
	\hat{m}_2(r)&=\begin{cases}
		0 & -\frac{3}{4}\le r\le \frac{3}{4}\\
		1 & \frac{3}{4}\le r\le \frac{7}{4}
	\end{cases}\notag
\end{align}
From (a), we know that $\mathbb{P}(\hat{m}_1(r)\neq m)$ is $\frac{1}{12}$ and $\mathbb{P}(\hat{m}_2(r)\neq m)$ is $\frac{1}{4}$.

\paragraph{(c)}
In this discrete case, r can only be in \{-1, 0, 1, 2\}. We can directly calculate the probability-of-error like (a):
\begin{align}
\mathbb{P}(\hat{m}(r)\neq m) &= \mathbb{P}(\hat{m}(r)=0| m=1)\mathbb{P}(m=1)+\mathbb{P}(\hat{m}(r)=1| m=0)\mathbb{P}(m=0) \notag\\
&= \frac{3}{4}(\frac{1}{8}\mathbb{I}[\hat{m}(0)=0]+\frac{3}{4}\mathbb{I}[\hat{m}(1)=0]+\frac{1}{8}\mathbb{I}[\hat{m}(2)=0])+\frac{1}{4}(\frac{1}{8}\mathbb{I}[\hat{m}(-1)=1]+\frac{3}{4}\mathbb{I}[\hat{m}(0)=1]+\frac{1}{8}\mathbb{I}[\hat{m}(1)=1])\notag\\
&= \frac{1}{32}(3\mathbb{I}[\hat{m}(2)=0]+\mathbb{I}[\hat{m}(-1)=1]+3\mathbb{I}[\hat{m}(0)=0]+6\mathbb{I}[\hat{m}(0)=1]+18\mathbb{I}[\hat{m}(1)=0]+\mathbb{I}[\hat{m}(1)=1])\notag
\end{align}
Therefore, $\hat{m}(r)$ should be
\begin{align}
	\hat{m}(r)=\begin{cases}
		0& r=-1,0\\
		1& r=1,2
	\end{cases}\notag
\end{align}
The associated probability of error is $\frac{1}{8}$.

\section{efficient frontier}
\paragraph{Proof:}
We write $P_D$ and $P_F$ as functions of $\eta$:
\begin{align}
	P_D(\eta)&=P_{L|H}(l\ge \eta|H=H_1)\notag\\
	P_F(\eta)&=P_{L|H}(l\ge \eta|H=H_0)\notag
\end{align}
where $L$ is the likelihood ratio. Therefore
\begin{align}
	\frac{\partial P_F(\eta)}{\partial \eta}&=\frac{\partial}{\partial \eta}\int^{\infty}_{\eta}p_{L|H}(l|H_0)dl\notag\\
	&=-p_{L|H}(\eta|H_0)\notag\\
	\frac{\partial P_D(\eta)}{\partial \eta}&=\frac{\partial}{\partial \eta}\int^{\infty}_{\eta}p_{L|H}(l|H_1)dl\notag\\
	&=-p_{L|H}(\eta|H_1)\notag\\
	&=-\int_{L(\mathbf{y})=\eta}p_{Y|H}(\mathbf{y}|H_1)dy\notag\\
	&=-\eta\int_{L(\mathbf{y})=\eta}p_{Y|H}(\mathbf{y}|H_0)dy\notag\\
	&=-\eta p_{L|H}(\eta|H_0)\notag
\end{align}
$P_D$ and $P_F$ form a parametric function on $\eta$, in which
\begin{align}
	\frac{\partial P_D}{\partial P_F}&=\frac{\frac{\partial P_D}{\partial \eta}}{\frac{\partial P_F}{\partial \eta}}=\eta
\end{align}
(b) has then been proven. To prove (a), we need to show that $\frac{\partial P_D}{\partial P_F}$ will not increase if $P_F$ increases.\\
For all $\eta_1, \eta_2$ satisfying $0\le \eta_1< \eta_2< \infty$, we have
\begin{align}
	P_F(\eta_1)-P_F(\eta_2)&=\int^{\infty}_{\eta_1}p_{L|H}(l|H_0)dl-\int^{\infty}_{\eta_2}p_{L|H}(l|H_0)dl\notag\\
	&=\int^{\eta_2}_{\eta_1}p_{L|H}(l|H_0)dl\ge 0\notag
\end{align}
Consider arbitrary $P_{F_1}$ and $P_{F_2}$ satisfying $0\le P_{F_1} < P_{F_2}\le 1$. If their corresponding parameters are $\eta_1$ and $\eta_2$, then we have $\eta_1\ge \eta_2$, since otherwise $P_F(\eta_1)-P_F(\eta_2)\ge 0$. Taking equation (1) into consideration, we have shown $\frac{\partial P_D}{\partial P_F}$ will not increase if $P_F$ increases.

\section{minimax test}
\paragraph{(a)}
The two posteriors $p_{Y|H}(y|H_0)$ and $p_{Y|H}(y|H_1)$, are formulated as
\begin{align}
	p_{Y|H}(y|H_0)&=\frac{1}{\sqrt{2\pi}\sigma}e^{-\frac{(y-s_0)^2}{2\sigma^2}}\notag\\
	p_{Y|H}(y|H_1)&=\frac{1}{\sqrt{2\pi}\sigma}e^{-\frac{(y-s_1)^2}{2\sigma^2}}\notag
\end{align}
Given priors $1-p$ and $p$, the likelihood ratio and the threshold $\eta$ are
\begin{align}
	L(y)&=\frac{p_{Y|H}(y|H_1)}{p_{Y|H}(y|H_0)}\notag\\
	&=e^{\frac{2(s_1-s_0)y+s_0^2-s_1^2}{2\sigma^2}}\notag\\
	\eta&=\frac{P_0}{P_1}\frac{C_{10}-C_{00}}{C_{01}-C_{11}}\notag\\
	&=\frac{1-p}{p}\notag
\end{align}
Therefore, the optimal Bayes' decision rule takes the form
\begin{align}
	y\mathop{\gtreqless}^{\hat{H}(y)=H_1}_{\hat{H}(y)=H_0}\frac{\sigma^2}{s_1-s_0}\ln\frac{1-p}{p}+\frac{s_0+s_1}{2} = \hat{y}
\end{align}
We define the right hand side of the above formula as $\hat{y}$. The cost of such decision is then
\begin{align}
	\varphi(\hat{H},p)&=(1-p)\mathbb{E}[C(H,\hat{H}(y))|H=H_0]+p\mathbb{E}[C(H,\hat{H}(y))|H=H_1]\notag\\
	&=(1-p)\int^{\infty}_{\hat{y}}\frac{1}{\sqrt{2\pi}\sigma}e^{-\frac{(y-s_0)^2}{2\sigma^2}}dy+p\int^{\hat{y}}_{-\infty}\frac{1}{\sqrt{2\pi}\sigma}e^{-\frac{(y-s_1)^2}{2\sigma^2}}dy\notag
\end{align}
Taking equation (2) into consideration, $\frac{\partial \varphi(\hat{H},p)}{\partial p}$ is derived as
\begin{align}
	\frac{\partial \varphi(\hat{H},p)}{\partial p}&=-\int_{\hat{y}}^{\infty}\frac{1}{\sqrt{2\pi}\sigma}e^{-\frac{(y-s_0)^2}{2\sigma^2}}dy+\frac{\sigma^2}{p(s_1-s_0)}\frac{1}{\sqrt{2\pi}\sigma}e^{-\frac{(\hat{y}-s_0)^2}{2\sigma^2}}\notag\\
	&\quad +\int^{\hat{y}}_{-\infty}\frac{1}{\sqrt{2\pi}\sigma}e^{-\frac{(y-s_1)^2}{2\sigma^2}}dy-\frac{\sigma^2}{(1-p)(s_1-s_0)}\frac{1}{\sqrt{2\pi}\sigma}e^{-\frac{(\hat{y}-s_1)^2}{2\sigma^2}}\notag\\
	&=\int^{\hat{y}}_{-\infty}\frac{1}{\sqrt{2\pi}\sigma}e^{-\frac{(y-s_1)^2}{2\sigma^2}}dy-\int_{\hat{y}}^{\infty}\frac{1}{\sqrt{2\pi}\sigma}e^{-\frac{(y-s_0)^2}{2\sigma^2}}dy+\frac{\sigma^2}{s_1-s_0}(\frac{p_{Y|H}(\hat{y}|H_0)}{P_1}-\frac{p_{Y|H}(\hat{y}|H_1)}{P_0})\notag\\
	&=\int^{\hat{y}}_{-\infty}\frac{1}{\sqrt{2\pi}\sigma}e^{-\frac{(y-s_1)^2}{2\sigma^2}}dy-\int_{\hat{y}}^{\infty}\frac{1}{\sqrt{2\pi}\sigma}e^{-\frac{(y-s_0)^2}{2\sigma^2}}dy\notag\\
	&=\int^{\hat{y}-s_1}_{-\infty}\frac{1}{\sqrt{2\pi}\sigma}e^{-\frac{t^2}{2\sigma^2}}dt-\int^{s_0-\hat{y}}_{-\infty}\frac{1}{\sqrt{2\pi}\sigma}e^{-\frac{t^2}{2\sigma^2}}dt\notag
\end{align}
If $p\le \frac{1}{2}$, then $\hat{y}\ge \frac{s_0+s_1}{2}$. Therefore $\frac{\partial \varphi(\hat{H},p)}{\partial p}\ge 0$, indicating that the cost increases as p increases. \\
If $p\ge \frac{1}{2}$, then $\hat{y}\le \frac{s_0+s_1}{2}$. Therefore $\frac{\partial \varphi(\hat{H},p)}{\partial p}\le 0$, indicating that the cost decreases as p increases.\\
In conclusion, the least favorable priors are $P_0=P_1=\frac{1}{2}$. The corresponding decision rule $\hat{H}_M(y)$ is
\begin{align}
	\hat{H}_M(y)=\begin{cases}
		H_0&y\le \frac{s_0+s_1}{2}\\
		H_1&y>\frac{s_0+s_1}{2}
	\end{cases}\notag
\end{align}

\paragraph{(b)}
The cost of the minimax decision rule $\hat{H}_M(y)$ is
\begin{align}
	\varphi(\hat{H}_M,p)&=\varphi(\hat{H}_M,\frac{1}{2})\notag\\
	&=\frac{1}{2}\int^{\infty}_{\frac{s_0+s_1}{2}}\frac{1}{\sqrt{2\pi}\sigma}e^{-\frac{(y-s_0)^2}{2\sigma^2}}dy+\frac{1}{2}\int^{\frac{s_0+s_1}{2}}_{-\infty}\frac{1}{\sqrt{2\pi}\sigma}e^{-\frac{(y-s_1)^2}{2\sigma^2}}dy\notag\\
	&=\int^{\frac{s_0-s_1}{2}}_{-\infty}\frac{1}{\sqrt{2\pi}\sigma}e^{-\frac{t^2}{2\sigma^2}}dt\notag
\end{align}
The expected cost of the optimum Bayes decision rule $\hat{H}_B(y)$ is
\begin{align}
	\varphi(\hat{H}_B,p)&=(1-p)\int^{\infty}_{\hat{y}}\frac{1}{\sqrt{2\pi}\sigma}e^{-\frac{(y-s_0)^2}{2\sigma^2}}dy+p\int^{\hat{y}}_{-\infty}\frac{1}{\sqrt{2\pi}\sigma}e^{-\frac{(y-s_1)^2}{2\sigma^2}}dy\notag
\end{align}
where $\hat{y}$ is given by equation (2). From (a) we already know that $\varphi(\hat{H}_B,p)\le \varphi(\hat{H}_M,\frac{1}{2})=\varphi(\hat{H}_M,p)$.\\
When $p=(e^2+1)^{-1}$, $s_0=0$, $s_1=2$, $\sigma^2=1$, the two costs are
\begin{align}
	\varphi(\hat{H}_M,p)&=\int^{-1}_{-\infty}\frac{1}{\sqrt{2\pi}}e^{-\frac{t^2}{2}}dt\notag\\
	&\approx 0.1587\notag \\
	\varphi(\hat{H}_B,p)&=\frac{e^2}{e^2+1}\int^{\infty}_{2}\frac{1}{\sqrt{2\pi}}e^{-\frac{(y-0)^2}{2}}dy+\frac{1}{e^2+1}\int^{2}_{-\infty}\frac{1}{\sqrt{2\pi}}e^{-\frac{(y-2)^2}{2}}dy\notag\\
	&\approx 0.0797\notag
\end{align}
\section{estimation}
\paragraph{(a)}
The distribution of y is 
\begin{align}
	p_y(y)&=\int^{\infty}_{0}p_x(x)p_{y|x}(y|x)dx\notag\\
	&=\int^{\infty}_{0}\frac{x}{a}e^{-x(y+\frac{1}{a})}dx\notag\\
	&=\frac{a}{(ay+1)^2}\notag
\end{align}
The Bayes' least-squares estimation of the case is
\begin{align}
	\hat{x}_{BLS}(y)&=\mathbb{E}[x|y]\notag\\
	&=\int^{\infty}_{0}xp_{x|y}(x|y)dx\notag\\
	&=\int^{\infty}_{0}x\frac{p_{y|x}(y|x)p_x(x)}{p_y(y)}dx\notag\\
	&=\int^{\infty}_{0}\frac{(ay+1)^2x^2}{a^2}e^{-x(y+\frac{1}{a})}dx\notag\\
	&=\frac{2a}{ay+1}\notag
\end{align}
The bias of the estimation is
\begin{align}
	b&=\mathbb{E}[\hat{x}_{BLS}(y)-x]\notag\\
	&=\int^{\infty}_{0}\int^{\infty}_{0}(\frac{2a}{ay+1}-x)\frac{x}{a}e^{-x(y+\frac{1}{a})}dxdy\notag\\
	&=\int^{\infty}_{0}0dy=0\notag
\end{align}
which is correct since it is a BLS estimator. The expected expectation $\mathbb{E}[\hat{x}_{BLS}(y)-x|y=y]$ of the belief is then 0(the above integration over x has already become 0) and the expected variance of the belief is
\begin{align}
	\lambda_{x|y}(y)&=\mathbb{E}[(\hat{x}_{BLS}(y)-x)^2|y=y]\notag\\
	&=\int^{\infty}_{0}(\frac{2a}{ay+1}-x)^2\frac{(ay+1)^2x}{a^2}e^{-x(y+\frac{1}{a})}dx\notag\\
	&=\frac{2a^2}{(ay+1)^2}\notag
\end{align}
The error variance of the estimator is
\begin{align}
	\lambda_{BLS}&=\mathbb{E}[(\hat{x}_{BLS}(y)-x)^2]\notag\\
	&=\int^{\infty}_{0}\int^{\infty}_{0}(\frac{2a}{ay+1}-x)^2\frac{x}{a}e^{-x(y+\frac{1}{a})}dxdy\notag\\
	&=\int^{\infty}_{0}\frac{2a^3}{(ay+1)^4}dy\notag\\
	&=\frac{2a^2}{3}\notag
\end{align}
Obviously we can also use $\lambda_{BLS}=\mathbb{E}[\lambda_{x|y}(y)]$ to get the error variance, which shares the last two lines of the above calculation.

\paragraph{(b)}
The posterior $p_{x|y}(x|y)$ is formulated as
\begin{align}
	p_{x|y}(x|y)=\frac{(ay+1)^2x}{a^2}e^{-x(y+\frac{1}{a})}\notag
\end{align}
We take the derivative over x, and obtain
\begin{align}
	\frac{\partial p_{x|y}(x|y)}{\partial x}=(\frac{(ay+1)^2}{a^2}-\frac{(ay+1)^3x}{a^3})e^{-x(y+\frac{1}{a})}\notag
\end{align}
When $x\le \frac{a}{ay+1}$, the partial derivative is non-negative, so the posterior increases as x increases. When $x\ge \frac{a}{ay+1}$, the posterior decreases as x decreases. Therefore, the MAP estimate of x based on observation of y is
\begin{align}
	\hat{x}_{MAP}(y)=\frac{a}{ay+1}\notag
\end{align}
The bias of the estimator is 
\begin{align}
	b&=\mathbb{E}[\hat{x}_{MAP}(y)-x]\notag\\
	&=\int^{\infty}_{0}\int^{\infty}_{0}(\frac{a}{ay+1}-x)\frac{x}{a}e^{-x(y+\frac{1}{a})}dxdy\notag\\
	&=-\int^{\infty}_{0}\frac{a^2}{(ay+1)^3}dy\notag\\
	&=-\frac{a}{2}\notag
\end{align}
The error variance for this estimator is then
\begin{align}
	\lambda_{MAP}&=\mathbb{E}[(\hat{x}_{MAP}(y)-x-b)^2]\notag\\
	&=\mathbb{E}[(\frac{a}{ay+1}-x+\frac{a}{2})^2]\notag\\
	&=\int^{\infty}_{0}\int^{\infty}_{0}(\frac{2a}{ay+1}-x+\frac{a}{2})^2\frac{x}{a}e^{-x(y+\frac{1}{a})}dxdy\notag\\
	&=\int^{\infty}_{0}\frac{2a^3}{(ay+1)^4}+\frac{a^3}{4(ay+1)^2}dy\notag\\
	&=\frac{11a^2}{12}\notag
\end{align}
\end{document}
